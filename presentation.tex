\documentclass{beamer}
\usepackage[russian]{babel}
\usepackage[utf8]{inputenc}
\usepackage[T1]{fontenc}
\usepackage{beamerthemesplit}
\usepackage{algorithm}
\usepackage{algpseudocode}
\usepackage{graphics,epsfig, subfigure}
\usepackage{url}

\definecolor{kugreen}{RGB}{50,93,61}
\definecolor{kugreenlys}{RGB}{132,158,139}
\definecolor{kugreenlyslys}{RGB}{173,190,177}
\definecolor{kugreenlyslyslys}{RGB}{214,223,216}
\setbeamercovered{transparent}
\mode<presentation>
{    \usetheme{Warsaw}
  \usecolortheme[named=kugreen]{structure}
\useinnertheme{circles}
\usefonttheme[onlymath]{serif}
\setbeamercovered{transparent}
\setbeamertemplate{blocks}[rounded][shadow=true]
}
%\setbeamertemplate{background}{\includegraphics[width=1\textwidth]{natfak_baggrund.pdf}}
% \logo{\includegraphics[width=1cm]{billeder/logo}}

\title{Разработка алгоритмов оптимизации групповой работы мобильных роботов}
\author{Студент группы с8503а, Спорышев М. С.}
\institute{Руководитель: \\ н.с. лаборатории необитаемых подводных аппаратов и их систем, к.т.н. Туфанов И. Е.}
\date{25 Июня 2015}


\begin{document}

\begin{frame}
\titlepage
\end{frame}


\begin{frame}{Термины}
\begin{itemize}
\item АНПА -- автономный необитаемый подводный аппарат
\item СПУ -- система программного управления
\end{itemize}
\end{frame}

\begin{frame}{Зачем нужны АНПА}

Обзорно поисковые задачи
\begin{itemize}
\item  Поиск затонувших объектов
\item Замеры параметров водной среды
\item Исследование локальных неоднородностей
\end{itemize}

Использование групп АНПА
\begin{itemize}
\item Централизованное управление
\item Групповое управление
\end{itemize}

\end{frame}

\begin{frame}{Используемая система управления}

\begin{enumerate}
\item Составление заданий для аппаратов
\item Распределение заданий меджу аппаратами
\begin{itemize}
    \item Выбор последовательности заданий
    \item Выбор варианта заданий
    \begin{itemize}
        \item Точка начала
        \item Точка конца
        \item Время выполнения
    \end{itemize}
\end{itemize}
\item Перепланирование
\begin{itemize}
\item Выход аппарат из строя
\item Появление нового аппарата
\item Появление новых заданий
\item Изменение заданий
\end{itemize}

\end{enumerate}

\end{frame}

\begin{frame}{Математическая модель}

\begin{itemize}
\item Имеется $m$ аппаратов и $n$ заданий. $q$-ый аппарат в начальный момент времени находится в точке $s_q$
\item $d_q(\mathbf{a}, \mathbf{b}) = |\mathbf{a} - \mathbf{b}| / u_q$ ~-- время перехода АНПА от точки $\mathbf{a}$ к точке $\mathbf{b}$, где $u_q$ - максимальная скорость $q$-го аппарата.


\item Планом аппарата названа последовательность пар $$p = ((i_1, j_1), (i_2, j_2), ..., (i_{|p|}, j_{|p|})), \forall k \in 1..|p|~ i_k \in 1..n, j_k \in 1..v_{i_k}$$.

\item Время выполнения аппаратом с номером $q$ плана $p$, составляет:
$$
t_q(p) = d_q(\mathbf{s}_q, \mathbf{a}_{i_1, j_1}) + \sum_{k=2}^{|p|} d_q(\mathbf{b}_{i_{k-1}, j_{k - 1}}, \mathbf{a}_{i_k, j_k}) + \sum_{k=1}^{|p|}l_{i_k, j_k}
$$

\item Общий план $P = (p_1, p_2, ..., p_m)$
\item Время выполнения общего плана: $t(P) = \displaystyle \max_{q \in 1..m} t_q(p_q)$

\item $
t(P) \rightarrow \min
$
\end{itemize}

\end{frame}

\begin{frame}{Существующие решения}
Множественная задача коммивояжера
\begin{itemize}
\item Каждое задание является единственной точкой
\item Полный граф
\item Стартовая вершина
\item Поиск опти мальной системы циклов
\item Минимизация суммы, максимума
\end{itemize}
\end{frame}

\begin{frame}{Жадный алгоритм}
% \begin{algorithm}
% \floatname{algorithm}{Алгоритм}
% \caption{Жадный алгоритм}\label{alg:greedy}
\begin{algorithmic}[1]

\ForAll {$t \in T$}
    \ForAll {$var \in Vars(t)$}
        \ForAll {$v \in V$}
            \State $time, pos \gets$ \Call{MinPathTime}{$t, var, Plan(v)}$
            \If {$time < minTime$}
                \State $minTime \gets time$
                \State $bestPos \gets pos$
                \State $bestV \gets v$
                \State $bestVar \gets var$
            \EndIf
        \EndFor
    \EndFor
    \State \Call{Insert}{$t, bestVar, Plan(bestV), bestPos$}
    \State \Call{OptimizeVars}{$bestV$}
\EndFor

\end{algorithmic}
% \end{algorithm}

\end{frame}

\begin{frame}{Оптимальный выбор варинатов}
Известна последовательность заданий, найти варианты заданий, дающие минимальное время выполнения такой последовательности.

Метод динамического программирования
\begin{itemize}
\item Номер последнего задания
\item Номер варианта задания
\end{itemize}

\begin{algorithmic}
\State $val \gets d[i - 1][var1] + Dist(p[i - 1], p[i])$
\If {$d[i][var2] > val$}
    \State $d[i][var2] \gets Min(d[i][var2], val)$
    \State $prev[i][var2] = var1$
\EndIf
\end{algorithmic}

\end{frame}

\begin{frame}{Генетический алгоритм}

\begin{itemize}
\item Решение представлено в виде двух хромосом
\begin{itemize}
\item Перестановка номеров заданий
\item Последовательность номеров аппаратов
\end{itemize}

\item 3 вида мутаций
\begin{itemize}
\item Swap
\item Reverse
\item Select
\end{itemize}

\item Скрещивание -- Partially matched crossover
\item Функция приспособленности
\end{itemize}

\end{frame}

\begin{frame}{Целочисленное программирование}
\begin{itemize}
\item Точное решение
\item TSP -- сотни заданий за секунды
\item Множественная задача коммивояжера
    \begin{itemize}
    \item Оптимизация максимума
    \item Бинарный поиск максимума и оптимизация суммы с доп. ограничениями
    \end{itemize}

\item Работает медленнее алгоритма Хельда-Карпа
\end{itemize}
\end{frame}

\begin{frame}{Реализация}


Стороннине библиотеки
\begin{itemize}
\item FlopC++
\item SYMPHONY
\item OpenCV
\item Boost
\end{itemize}

Python
\begin{itemize}
\item Pandas
\item matplotlib
\end{itemize}

\end{frame}

\begin{frame}{Тестирование}

\end{frame}

\begin{frame}{Результат}
Алгоритмы успешно внедрены
\begin{itemize}
\item до 18 заданий -- алгоритм Хельда-Карпа
\item до 40 заданий -- генетический алгоритм
\item от 40 заданий -- жадный алгоритм
\end{itemize}
\end{frame}

\begin{frame}{Заключение}
\begin{itemize}
\item Система контроля версий git, 52 коммита, +4900 -1800. Языки C++ и Python.
\end{itemize}

\end{frame}

\end{document}